\section{Introduction}\label{GameIntro}\thispagestyle{SectionFirstPage} % Hide headers on the first page of the section
\lhead{Introduction to Bazar Blot}
\subsection{The Origin of the Game}
\hspace{\parindent} Belote is a 32-card game played primarily in France and in some other countries like Armenia, Belgium, Bulgaria, Croatia, Cyprus, and Georgia.
In each country, the game has encountered some modifications; however, the fundamental rules are mostly identical everywhere.
Firstly, it would be more accurate to introduce the original game, Belote, and then add the Armenian modifications to get the overall understanding of Bazar Blot's rules and gameplay.
The game appeared around 1900 in France and is one of the most popular card games in France and other European countries.
The rules were first published in French in 1921.
The name of the game \textit{``Belote''} has possibly originated from the term used to describe a pair of \textit{King} and \textit{Queen} of a trump suit, which gives 2 bonus points to the pair of players when one of them has this combination.
This name also varies from region to region.
In Saudi Arabia, it is called \textit{Baloot}, in Cyprus, the name is \textit{Pilotta}, and in Armenia, it is known as \textit{Blot}.

\subsection{General Rules}
\hspace{\parindent} Belote is played with a \textit{Piquet} deck, which is a standard deck stripped from 2 to 6's, meaning it has 4 suits: \textit{Spades, Hearts, Clubs, Diamonds} and 8 ranks: \textit{A, K, Q, J, 10, 9, 8, 7}.
The game is played with 2 teams of 2 players and playing in the turn counterclockwise.
The deck is typically shuffled and offered the player preceding the dealer to cut the deck.
The cutter may cut or just tap on the top of the pack in which case no cutting is done.
The first dealing is done by the winners of the previous game, which in Armenia sometimes is called ``Pativ tal''.
The cards are dealt counterclockwise, starting from the dealer's successor.
In the original version, each player receives a pack of 3 cards, then another set of 2.
The remains faced down until the contract is being agreed.
The remaining cards are dealt after the bidding – a group of three for each player except the player who got the card that was in the middle, who gets two.
In the Armenian \textit{``Bazar Blot''} version, each player gets a pack of 4 cards, then another equal pack, after which the bidding process starts.

\subsection{Bidding}
The possible contracts for \textit{``Bazar Blot''} are:
\begin{itemize}
    \item Clubs $\clubsuit$
    \item Diamonds $\diamondsuit$
    \item Hearts $\heartsuit$
    \item Spades $\spadesuit$
    \item No Trumps
\end{itemize}
Every player must either suggest a higher bid or:
\begin{itemize}
    \item Pass
    \item Double (Coinchee or contra), if they are sure the opponents will not be able to get that many points.
    \item Re-Double (Re-contra), if the other team have doubled bidder's or bidder partner's contract.
    \item Capot, meaning that team will win the round without giving any points to the opponents.
\end{itemize}

The bidding procedure is over when one of the following happens:
\begin{itemize}
    \item Four \textit{passes} are announced.
    \item A contract is \textit{Doubled} and (and not) \textit{Re-Doubled}
\end{itemize}
\subsection{Declarations}

\hspace{\parindent} The \textit{declarations} are special combinations of cards that need to be announced during the first trick and be shown to the rest of the players during the second trick to get bonus points for them.
The exception is \textit{Belote/Re-Belote}, which is not being announced during the first trick, but the player needs to announce while playing either \textit{King} or \textit{Queen} of trump suit.
If any of these rules are being broken the \textit{declarations} are being omitted.
There are 5 types of such combinations:
\begin{itemize}
    \item Tierce - a sequence of three cards of the same suite - 20 points
    \item Fifty - a sequence of four cards of the same suite - 50 points
    \item Hundred - a sequence of five cards of the same suite - 100 points
    \item Belote/Re-Belote - a pair of \textit{King} and \textit{Queen} of the trump suit -  20 points
    \item Four of a Kind - all four cards of the same rank of all suits - points differ with the contract (see the table below)
\end{itemize}

\begin{center}
    \begin{tabular}{|c|c|c|}
        \hline
        \textbf{Card} & \textbf{Trump}  & \textbf{No-Trump}\\
        \hline
        A & 11 & 19\\
        \hline
        K & 10 & 10\\
        \hline
        Q & 10  & 10\\
        \hline
        J & 20 & 10\\
        \hline
        10 & 10 & 10\\
        \hline
        9 & 14 & 0\\
        \hline
        8 & 0 & 0\\
        \hline
        7 & 0 & 0\\
        \hline
    \end{tabular}
\end{center}
\textit{Note that the sequences are in the ``A K Q J 10 9 8 7'' order of the same suit}.\\
It is also important that \textbf{one card can only participate in at most one declaration}.

Furthermore, if more than 1 players have any of these combinations, they can either sum them up and get credits for each of them if they are teammates or if they are opponents the one that has a larger combination voids the smaller one.\\
\textit{Note: Tierce$<$50$<$100$<$Four of a Kind}\\
In case if 2 opponents have the same combination, the one with a top rank of combination voids the other.\\
\textit{E.g. Tierce(A,K,Q) $<$ Tierce(K,Q,J)}\\
The only combination that cannot be voided is \textit{Belote/Re-Belote}

\subsection{Playing}
\hspace{\parindent} The dealer's successor(the player to the right of the dealer) plays their card first.
The first player can play any card; however, the subsequent players must follow the suit if they can.
If they do not have a card of the same suit, they must play a trump card.
If they cannot play trump either, they can play any suit.
If the first card played is trump, the subsequent players must follow the suit as well as play a trump card that beats all the cards on the table if they can.
It is called \textit{``raising''} and applies only for trump suits.
If they cannot play such card, they put any trump card.
In case they do not have any trump, they can play any suit.
If non-trump card was played first, and then trump, the subsequent player must follow the suit of the first card, and does not have to put a higher trump; however, if the player on turn does not have a card that follows the suit of the first card, they have to put a higher trump.
If any of the above rules are being broken, the opponents get 162 points in addition to all \textit{``declarations''} declared in the round regardless of who has made the declaration and current round ends.
When each player has played 1 card, the player whose card was the highes wins the \textit{``hand''} or \textit{``trick''} and collects the cards played in this trick and plays first in next hand/trick.


\subsection{Scoring}
\hspace{\parindent} Every team counts the points on the tricks they have won together.
The winner of the last trick adds 10 to the overall points.\\

Each card has a specific value, depending on the contract.
\begin{center}
    \begin{tabular}{|c|c|c|c|}
    \hline
    \textbf{Card} & \textbf{Trump} & \textbf{Regular} & \textbf{No-Trumps}\\
    \hline
    A & 11 & 11 & 19\\
    \hline
    K & 4 & 4 & 4\\
    \hline
    Q & 3 & 3 & 3\\
    \hline
    J & 20 & 2 & 2\\
    \hline
    10 & 10 & 10 & 10\\
    \hline
    9 & 14 & 0 & 0\\
    \hline
    8 & 0 & 0 & 0\\
    \hline
    7 & 0 & 0 & 0\\
    \hline
\end{tabular}
\end{center}
If the contract for the round has been agreed to be any of the suits, then the scoring is being calculated with the \textit{Trump} table, and for the rest of the suits by the \textit{Regular} table.
For the case, if the contract has been agreed to be \textit{No-Trumps}, the calculations are done using the \textit{No-Trumps} table.\\

The overall score is divided by ten and rounded.
The rounding is done differently.
If usually we take the limit of rounding to be 5, in Bazar Blot this limit is 6.
If the result of division and rounding for the team that has made the final bid is greater than or equal to the bid, the bidding team sums up the points received from the round to the bid and adds to the overall score of the game, and is said to be the winner of the round.
The opponent team gets $(16 - (\textit{the score of bidding team}))$ points.
Otherwise, the opponent team gets 16 points in addition to the bid, and the bidding team is said to lose the round and gets no points.
Special cases are \textit{Capot}, \textit{Contra/Re-contra}, and \textit{No-Trumps}.
If the contract has been agreed to be \textit{No-Trumps}, the bid is being multiplied by 2 and summed to the overall points for the round for the winning team.
If the team fails to get necessary points, the opposing team gets $(16 + 2 \times bid)$ points for the round.
If any of the teams has done \textit{Capot} regardless the bid, they get $(25+bid)$ points.
If the contract has been \textit{Doubled(Contra)}, the winning team gets $(2\times bid + 16)$ points.
If the contract has been \textit{Doubled(Contra)} and \textit{Re-Doubled(Re-Contra)} the winning team gets $(4\times bid + 16)$ points.
Each team add the points retrieved from \textit{declarations} to the overall score.

\subsection{Environment Type}
\begin{itemize}
    \item Partially Observable (Opponents' cards are not visible)
    \item Multi-Agent (4 players)
    \item Stochastic (The deck is shuffled after each round)
    \item Sequential (Actions are interconnected over time)
    \item Static (No changes over time)
    \item Discrete (Finite amount of actions)
    \item Known (Rules are fully known)
\end{itemize}